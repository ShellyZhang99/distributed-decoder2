% Chapter 1

\chapter{绪论}

\section{研究背景及意义}

嗅探技术(Profiling)是一种被广泛应用的技术,它通过获取程序的运行时信息,能够分析程序的运行过程,并被应用到程序开发的各个阶段, 包括调试、测试、问题修复、性能分析、日志等。通过Profiling,开发人员可以了解到程序运行过程中在哪些部分花费更多的时间,程序运行过程的函数调用图等信息,利用程序运行时得到的Profiling信息,一方面,通过分析可以找到比预期执行慢的程序,从而进行优化并使其获得更快的执行效率,另一方面,通过分析出函数调用次数与预期的比例,可以发现程序的潜在错误。\upcite{GNUGprof}

通常来讲,可以将Profiling分为两大类:基于插桩(Instrumentation)实现的软件Profiling以及借助底层实现的硬件Profiling。软件Profiling在软件层面通过程序插桩技术,在原程序的基本块中注入额外的用于收集执行信息的输出代码,将源程序重新编译后从而在程序运行时获取程序执行过程中的控制流、数据流等信息,用以分析程序的执行过程。\upcite{OptimallyProfilingandTracingPrograms}

gprof和gcov是GNU的Profiling工具实现,gprof能够较为精确的提供函数级别的Profiling信息,评估程序在每一部分的执行时间,然而这类信息较为有限,gcov则通过编译前的插桩,能提供程序的行级别的Profiling信息,分析每行代码的执行频率,获得测试代码覆盖率等信息。\upcite{gcc}\upcite{gprof-a-call-graph-execution-profiler}
然而基于软件方式的Profiling通常是侵入式的,额外代码的引入一方面可能会影响程序的正常执行,另一方面要获取精准信息必然会对运行时程序带来较大的额外开销(插桩后的程序执行时间可能会比原来要慢50\%以上),这给对实际应用程序的Profiling造成了极大的困难。\upcite{Hardware-Based-Profiling}于此相比,基于硬件机制的Profiling可以借助已有的CPU硬件相应模块,实现在程序运行底层对程序执行信息的收集与跟踪,硬件级别的Profiling是非侵入式的,引入的额外开销较低,具有较高的研究价值。

数字连续Profiling基础架构是一种基于采样的Profiling系统,它通过对处理器提供的程序计数器的采样进行数据收集和分析,支持对生产应用程序的连续Profiling,基于对采样信息的分析,它能够确定指令停顿的原因,例如高速缓存未命中、资源争用、分支错误预测等,提供细粒度的指令级Profiling,指导用户和自动优化器找出性能问题的潜在原因,并提供解决问题的可能思路。\upcite{gprof-a-call-graph-execution-profiler} 

最后分支记录(Last Branch Record, LBR)和分支追踪存储(Branch Trace Store, BTS)是Intel处理器提供的硬件Profiling机制,LBR能够记录最后若干个分支记录到一组特定于模型的寄存器(MSR)中,利用LBR可以对分支执行进行采样,每次采样能够获取到最后8至32个分支记录信息,然而由于LBR只能记录最后的若干个分支,能够获取的Profiling信息十分有限,BTS能提供更多分支信息的记录,它将分支信息保存在缓冲区中,并在缓冲区满时产生中断信号导出到磁盘中,然而BTS的引入会给程序的执行造成巨大的额外开销,同时对每个分支记录的24字节信息使得BTS的记录信息十分庞大,这使得它并没有被广泛使用。\upcite{LBR}

Intel处理器追踪(Intel Processor Trace, Intel PT)是一种在Intel处理器上支持硬件Profiling的新硬件机制,从第五代核处理器BROADWELL开始,Intel处理器提供了对于Intel PT的支持,并在接下来的处理器中提供了普遍的支持。与BTS相比,Intel PT支持以更低的额外开销追踪,并且能够收集时间等额外信息,同时对于不同的分支,Intel PT对于追踪数据的生成采取了各种不同形式的压缩方式,生成更简洁的记录信息。\upcite{processor-tracing}\upcite{AddingProcessorTracesupporttoLinux}

Intel PT通过追踪CPU上的分支执行,能够重构代码执行的整个控制流。通过Intel PT实现硬件Profiling主要包含两个过程:(1)Intel PT追踪程序运行时分支跳转信息,产生一系列的数据包并压缩到二进制数据流中;(2)解码器以可执行文件和收集得到的压缩数据流为输入,进行解码从而重新构建程序执行时完整的控制流。\upcite{Intel}

基于Intel PT的硬件Profiling降低了对源程序执行产生的影响,减少了追踪过程产生的开销,但实际上将开销转移到了解码阶段。追踪过程中产生较多的追踪信息(每个核每秒成百兆字节)通常使得解码阶段的开销比追踪阶段慢几个数量级。提高Intel PT的解码效率,减少Intel PT解码阶段的开销,对于基于Intel PT的硬件Profiling具有重要的意义。

\section{研究现状}
Intel在引入Intel PT的新特性后受到了广泛的关注,它通过在硬件层面收集的大量的控制流等程序运行时信息,通过离线的解码重新构建完整的程序执行信息,在调试、测试、程序修复、性能优化等问题上都可能会有着重要的用途。不过,由于Intel提供Intel PT硬件机制时间不长,现在关于Intel PT的很多研究和使用仍然处在初始阶段。

为解决生产过程中出现的故障在测试中难以重现的问题,Failure Sketching利用Intel PT的控制流追踪,结合程序静态分析和程序动态分析策略,实现了一个能够自动化提供生产过程错误产生的根本原因的调试工具,该工具能够以较小的开销给出导致程序故障根本原因的语句,显示程序成功和失败运行的区别。\upcite{FailureSketching}

在GRIFFIN中,作者发现利用具有类似于Intel PT能够重新构建程序执行流的特性的硬件辅助控制流完整性(Control Flow Integrity, CFI)实施系统具有较高的效率和较好的灵活性,并基于Intel PT设计了一种并行化的方法实施各种类型CFI策略,在Linux 4.2内核中GRIFFIN能够在包括火狐浏览器及其即时编译的代码等多种软件得到完整的CFI策略实施,比基于软件的插桩具有更低的额外开销。\upcite{Griffin}

将Intel PT的追踪信息应用于硬件延迟的有效探查、追踪,和系统调用的准确分析也得到了比较好的结果,相比于基于插桩的Profiling方式,它对程序性能的影响大大降低,平均只有2\%至3\%,同时能够获得更为精确的分析信息。\upcite{Hardware-assisted-instruction-profiling}

REPT是利用Intel PT实现了一个能够对于部署系统的软件故障进行反向调试的系统,通过对Intel PT追踪数据的解码与对二进制文件的静态分析,REPT能够准确地重建程序的执行历史,集成到Windows的调试工具并对16个漏洞进行评估,实验表明,它能够有效和精确的还原这些漏洞的数据值。\upcite{REPT}

这些工具的实现与应用基于Intel PT,因此都离不开对于Intel PT追踪数据的解码,Intel PT在追踪过程中将信息以压缩数据包的形式收集到二进制流中,只有通过对PT数据的解码才能够获取我们所需要的程序执行时信息。目前提供对Intel PT数据进行解码的工具主要有:LIBIPT(Intel PT Decoder Library)是Intel对于Intel PT解码器的官方参考实现,它可以作为一个独立的库被调用或者部分或全部的集成到其他工具中,LIBIPT提供了解码过程中不同层次的抽象,通过调用LIBIPT提供的函数接口,可以实现对Intel PT追踪数据的解码。\upcite{Intel} \upcite{libipt} SimplePT是Andi Kleen实现的Linux内核模块,支持在Linux系统上收集和导出Intel PT的信息(不支持PT数据的连续导出,只能保存程序结束前的部分PT数据到磁盘),解码器基于LIBIPT实现,解码结果仅包括函数级别的信息。\upcite{simple-pt} Perf是Linux系统自带的性能分析工具,从内核4.1版本开始,可以通过perf\_event系统调用接口配置Intel PT收集信息,从内核4.3版本开始,perf添加了对Intel PT事件的支持,且支持持续PT数据的导出,perf工具的解码结果提供了指令级Profiling信息,它包括程序执行时每个分支的方向。\upcite{perf}不过,现有工具对于Intel PT的解码过程均是顺序执行的,因此解码会花费较多的时间,为减小Intel PT解码过程开销,有必要实现一个并行化的Intel PT解码器。

\section{研究内容}
本课题主要研究内容为在利用Intel PT进行硬件层面的Profiling时,减少解码阶段的开销,主要考虑通过解码阶段的并行化来进行实现。本课题利用Intel PT实现了一个简单的C/C++程序Profiling的工具,能够提供源程序函数、行两个级别的Profiling信息,包括源程序每个函数被调用的次数,每一行被执行的次数,主要工作包括三个方面的内容:1.在Linux 5.3内核下,利用Linux提供的系统调用perf event内核接口配置Intel PT进行PT数据的捕获,并能够连续向磁盘中导出捕获的数据;2.基于1中捕获的数据和程序可执行文件进行解码,重构原始程序在机器指令级别的完整执行流,在解码的过程中实现并行化,对Intel PT追踪数据进行划分,对划分后的数据进行并行的解码,从而减少解码阶段的开销,提高解码的效率;3.利用源程序编译(编译器为gcc/g++,编译选项需添加-g参数)阶段的调试信息实现解码后的机器指令到C/C++源程序的映射,从而得到程序执行过程中在源程序行、函数两个级别的Profiling信息。

对Intel PT的解码阶段进行并行化设计主要思路是首先借用Intel官方解码参考实现LIBIPT的思想,并在在其基础上进行改进,对Intel PT追踪数据的解码进行不同层次的封装,能在数据包层、事件层、指令层三个不同的层次对收集的Intel PT数据进行解码;然后,通过Intel PT数据包层的解码器对Intel PT追踪数据进行较为平均的划分,对划分的每一段Intel PT数据分别分配不同的指令层解码器进行独立的解码,实现解码阶段的并行化,同时在从机器指令到源程序映射的过程中,采用同样的并行策略,对每一段Intel PT数据解码得到的机器指令进行并行化的到源程序的映射。实验中追踪和解码了SIR(Software-Artifact Infrastructure Repository)中的11个C/C++程序,分析评估了追踪过程的额外开销,比较了并行化解码器与普通解码器的效率,同时在最后给出了本课题实现的C/C++程序Profiling工具在源码上的输出结果。

\section{论文组织结构}
本论文在第一章概述了Profiling相关的背景知识和利用Intel PT进行硬件Profiling的优势,简介了Intel PT的研究现状和对追踪数据并行化解码的意义,并给出了课题的主要研究点和论文组织结构。

第二章简要介绍了Intel PT相关的知识,给出了利用Intel PT进行硬件Profiling的流程和常用的几种PT数据包形式与作用以及利用Intel PT进行程序追踪的几种方法(具体介绍了本课题使用的perf event系统调用)。

第三章是基于Intel PT并行化解码器的设计方案和算法实现,首先具体介绍了数据包、事件、指令层三个级别解码器的具体功能和调用方法,然后给出了利用数据包、指令两层解码器进行并行化解码的方案和算法实现。 

第四章是实验部分,追踪和解码了SIR提供的10个C/C++程序,分析了追踪过程开销和解码过程效率,最后介绍了利用编译阶段生成的调试信息实现机器指令到源程序映射的方法,给出了实现的Profiling工具的最终结果。

第五章给出了本文的总结与展望。