% Chapter 5

\chapter{总结与展望}
本课题实现了一个基于Intel PT的C/C++程序的Profiling工具,能够给出C/C++程序在源码层面的Profiling信息,包括源码的每一行的执行次数,每个函数的调用次数,并且能够给出函数调用层次关系。这些Profiling信息能够很好的用于程序的性能优化、调试等等方面。在对Intel PT追踪数据进行解码时,实现了解码过程的并行化,缩短了解码过程的开销,提高了解码的效率,并通过实验分析了SIR中的10个C/C++程序的追踪过程开销和解码的效率。

但是,尽管利用Intel PT进行硬件层面的Profiling在追踪阶段只会引入很低的额外开销,追踪过程中记录程序控制流会产生大量的数据,并需要在追踪过程中将其导出到磁盘,由于CPU执行的速度远大于磁盘I/O速度,将PT数据导出到磁盘时,可能会产生数据缺失,这会造成最终Profiling信息的不完整;由于编译期的各种优化措施,机器指令到源程序的映射具有一定的误差,这可能会造成最终Profiling信息的不准确。因此更好地提高Profiling信息的完整性和准确性是需要继续研究的问题。此外,Intel PT追踪数据解码后能够重构程序完整的执行流,因此能提供非常丰富的Profiling信息,此次毕设仅输出了函数调用层次结构以及源码级别函数、行的执行次数,以更好的形式输出得到的Profiling信息也是一个需要完善的部分。